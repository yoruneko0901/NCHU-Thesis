\documentclass[12pt,a4paper]{report}
\usepackage{fontspec} % 加載 fontspec 套件以使用系統字體
\setmainfont{Microsoft JhengHei UI} % 設定主要字體為微軟正黑體
% \setmainfont{標楷體} % 設定主要字體為標楷體(xe)
% \setmainfont{新細明體} % 設定主要字體為新細明體(xe)
% \setmainfont{GenSenRounded JP R}(lua)
\usepackage[utf8]{inputenc} % 支援UTF-8編碼
\usepackage{geometry} % 控制頁面邊界
\geometry{a4paper, margin=1in} % 設定A4紙張,邊界為1英寸

\usepackage{setspace} % 用來調整行距
\onehalfspacing

\usepackage{fancyhdr} % 控制頁眉與頁腳
\pagestyle{fancy}
\setlength{\headheight}{14.5pt}
\fancyfoot[C]{\thepage} % 頁腳中間顯示頁碼

\usepackage{titlesec} % 控制章節標題的間距
\titleformat{\chapter}[display]
 {\normalfont\huge\bfseries}
 {\chaptertitlename\ \thechapter}
 {20pt}
 {\raggedright\Huge}
\titlespacing*{\chapter}{0pt}{0pt}{40pt} % 調整章節標題的上方和下方留白

% 封面格式
\begin{document}

\begin{titlepage}
    \centering
    \vspace*{1in}

    {\LARGE \textbf{論文標題}}\\[2cm]

    \textbf{作者名稱}\\
    學號:12345678\\
    系所名稱\\
    學校名稱\\[2cm]

    \vfill

    \textbf{指導教授:XXX教授}\\[3cm]

    \today\\

    \vspace*{1cm}
\end{titlepage}

\clearpage % 封面與正文分開

\pagenumbering{arabic} % 開始頁碼編號


\chapter*{簡介} % 無編號章節標題
\vspace{-1cm} % 調整上方空白,-1cm 可根據需求調整
\addcontentsline{toc}{chapter}{簡介} % 手動添加到目錄
這是一個範例的LaTeX文件,展示了如何在LaTeX中顯示中文。

\chapter*{研究背景} % 無編號章節標題
\vspace{-1cm} % 調整上方空白
\addcontentsline{toc}{chapter}{研究背景} % 手動添加到目錄
這個章節描述了研究的背景、問題的來源及其重要性。

\chapter*{文獻回顧} % 無編號章節標題
\vspace{-1cm} % 調整上方空白
\addcontentsline{toc}{chapter}{文獻回顧} % 手動添加到目錄
在這個章節中,將回顧與本研究相關的文獻和研究成果。

\chapter*{研究方法} % 無編號章節標題
\vspace{-1cm} % 調整上方空白
\addcontentsline{toc}{chapter}{研究方法} % 手動添加到目錄
此章節介紹本研究所使用的方法、流程和技術。

\chapter*{結果與討論} % 無編號章節標題
\vspace{-1cm} % 調整上方空白
\addcontentsline{toc}{chapter}{結果與討論} % 手動添加到目錄
在這裡呈現研究的結果並進行分析和討論。

\chapter*{結論與建議} % 無編號章節標題
\vspace{-1cm} % 調整上方空白
\addcontentsline{toc}{chapter}{結論與建議} % 手動添加到目錄
最後,總結研究結果,並提供後續研究的建議。

\chapter*{參考文獻} % 無編號章節標題
\vspace{-1cm} % 調整上方空白
\addcontentsline{toc}{chapter}{參考文獻} % 手動添加到目錄
在這個章節中列出所有引用的文獻。


\end{document}
