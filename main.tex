\documentclass[12pt,a4paper]{report}
\usepackage{fontspec}
\setmainfont{Microsoft JhengHei UI} % 设定中文字体为微軟正黑體
\usepackage[utf8]{inputenc}
\usepackage{geometry}
\geometry{a4paper, left=3cm, right=3cm, top=3cm, bottom=3cm} % 设定页面边距
\usepackage{setspace}
\onehalfspacing % 1.5倍行距
\usepackage{fancyhdr}
\pagestyle{fancy}
\fancyhf{}
\fancyfoot[C]{\thepage}
\usepackage{titlesec}
\titlespacing*{\chapter}{0pt}{-20pt}{20pt} % 减少章节标题上方的空白
\usepackage{graphicx}
\usepackage{tocloft}

\begin{document}

% 封面
\begin{center}
    {\LARGE
        國立中興大學XXXX學系\\
        X士學位論文
    }
    \vspace{4.8cm}

    {\huge
        (中   文   論   文   題   目)
    }
    \vspace{2cm}

    {\LARGE
        (英   文   論   文   題   目)
    }
    \vspace{6.2cm}

    {\LARGE
        \begin{tabular}{lr}
          指導教授: & XXX \\
          研\hspace{0.33cm}究\hspace{0.33cm}生:& XXX
        \end{tabular}
    }
    \vspace{2.9cm}

    {\LARGE
        中華民國XX年XX月
    }
\end{center}

% 空白页
\newpage
\thispagestyle{empty}
\mbox{}

% 中文书名页


% 致謝辞
\newpage
\chapter*{致謝辞}
\addcontentsline{toc}{chapter}{致謝辞}
在此,我要感谢所有在我研究期间提供帮助的人,包括我的导师XXX教授,我的家人,以及所有支持我的朋友和同事。

% 中文摘要
\newpage
\chapter*{中文摘要}
\addcontentsline{toc}{chapter}{中文摘要}
本研究探討了...(摘要内容)。

\newpage
% 英文摘要
\chapter*{SUMMARY}
\addcontentsline{toc}{chapter}{SUMMARY}
This study investigates...(英文摘要内容)。

% 目次
\newpage
\tableofcontents

% 正文示例
\newpage
\chapter*{第一章 簡介}
\addcontentsline{toc}{chapter}{第一章 簡介}
本章將概述研究背景、研究目的及研究方法。

\chapter*{第二章 文獻回顧}
\addcontentsline{toc}{chapter}{第二章 文獻回顧}
本章回顧了相關的研究文獻,並討論了與本研究相關的理論基礎。

% 继续添加章节...

\newpage
% 参考文献
\chapter*{參考文獻}
\addcontentsline{toc}{chapter}{參考文獻}
\begin{itemize}
    \item [1] 吳聰敏, 吳聰慧 (2005). 《cwTEX 排版系統》, 台北。
    \item [2] Bringhurst, Robert (1996). The Elements of Typographic Style, Vancouver: Hartley  Marks, 2nd edition.
\end{itemize}

% 附錄示例
\newpage
\chapter*{附錄}
\addcontentsline{toc}{chapter}{附錄}
附錄内容示例。

\end{document}
